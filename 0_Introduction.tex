\section{Introduction}
% General Introduction parts
\noindent
Cancer is one of the leading causes of death worldwide with around 20 million cases reported in 2022 - corresponding to around one in five men or women developing cancer in their lifetime \cite{bray_global_2024}. The conventional method for cancer diagnosis includes taking a biopsy of suspect tissue from a patient during surgery, and then subsequent pathological analysis. This pathological analysis generally involves staining of the biopsy with haematoxylin and eosin (HE). The haematoxylin stains nucleic acids a deep blue-purple colour, whilst the eosin is pink and non-specifically stains proteins \cite{fischer_hematoxylin_2008}. These HE stained samples can then be used by a pathologist to perform diagnosis. This process is manual, error-prone, subjective, costly, and time-consuming; often requiring transport to a laboratory for processing by highly trained technicians \cite{hollon_near_2020}. The time between pathologists receiving samples and a diagnosis being returned to a surgeon is often around 20 minutes \cite{novis_interinstitutional_1997} - any reduction in this processing time would therefore help to guide surgeries more effectively.\\

New methods are being developed to expedite the cancer diagnosis process by using machine learning in conjunction with a range of imaging technologies including stimulated Raman spectroscopy (SRS) \cite{hollon_near_2020, sarri_fast_2019, kondepudi_foundation_2024, jiang_opensrh_2022}, second harmonic generation microscopy (SHG) \cite{sarri_fast_2019}, and Fourier transform infrared spectroscopy (FTIR) \cite{tomas_detection_2022, berisha_deep_2019}. This project will focus on ........
\subsection{Future Vision}
\subsection{Novelty of This Work}